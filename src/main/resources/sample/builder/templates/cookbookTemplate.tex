\documentclass[10pt,a4paper]{article}
\usepackage[utf8x]{inputenc}
\usepackage{makeidx}
\usepackage{graphicx}
\usepackage[left=2cm,right=2cm,top=2cm,bottom=2cm]{geometry}
\usepackage{scrextend} %\ifoddpage
\usepackage{changepage}
\usepackage{fancyhdr} %Kopf- und Fußzeile
\pagestyle{fancy}
\fancyhf{}
\usepackage[encoding,filenameencoding=utf8x]{grffile}
%Kopfzeile links bzw. innen
%\fancyhead[L]{Kai Nortmann \& Christian Zöller}
%Kopfzeile rechts bzw. außen
%\fancyhead[R]{\today}
%Linie oben
%Fußzeile mittig
%\fancyfoot[c]{Labor 1 RS-232-Schnittstelle}
%Fußzeile gerade rechts

\renewcommand{\footrulewidth}{0.5pt} %Linie unten
\begin{document}


    #foreach( $recipe in $cookbook.getRecipes() )

        \fancyhf{}

        #set ($refNum = $refNumList.get($recipe))
        #set ($index = $refNum.indexOf('.'))
        #if($index < 0)
            #set($index = $refNum.length()) ##.length needs to be tested with only 1 sortlevel
        #end

        \chead{$refNum.substring(0,$index)}
        \section*{$recipe.getTitle()}
	    \begin{minipage}[t]{0.4\textwidth}
	        \subsection*{Zutaten}
	        \begin{itemize}
                #foreach( $ingredient in $recipe.getIngredients())
                    \item {$ingredient.getLeft().getName() $ingredient.getMiddle() $ingredient.getRight().getName()}
                #end
	        \end{itemize}
	    \end{minipage}
	    \hfill
        \begin{minipage}[t]{0.4\textwidth}
            \vspace{-2ex}
            \includegraphics[width=\linewidth]{"${imgDir}/${recipe.getTitle()}${recipe.getID()}"}
        \end{minipage}
        \subsection*{Zubereitung}
        $recipe.getText()\\

        \checkoddpage
        \ifoddpage
            \lfoot{$refNum}
        	\newpage{}
       	    \thispagestyle{empty}
     	    \mbox{}
     	    \newpage
       	\else
       	    \rfoot{$refNum}
       		\newpage{}
        \fi
    #end
\end{document}
